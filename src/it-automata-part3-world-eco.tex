% https://habr.com/ru/companies/ruvds/articles/574352/
% https://guides.nyu.edu/LaTeX/sample-document
% https://ru.stackoverflow.com/questions/222769/latex-как-быть-с-русским-текстом/757767#757767

\documentclass{article}
% без этой строчки (модуль cmap) не будет работать поиск внутри документа и копирование русского текста:
\usepackage{cmap}
\usepackage[utf8]{inputenc}
\usepackage[english,russian]{babel}

% разбить длинные адреса URL, которые не умещаются в строку
% https://tex.stackexchange.com/questions/54946/how-to-break-a-long-url
\usepackage[hyphenbreaks]{breakurl}
\usepackage[hyphens]{url}

% слова с дефисом (здесь в основном - "аппаратно-программная") по умолчанию не переносятся автоматом и вылезают за пределы строки
% чтобы этого не происходило, следует заменить дефис на команду "=
% https://www.linux.org.ru/forum/general/11119398
% http://www.inp.nsk.su/~baldin/LaTeX/ctex.pdf
% https://stackoverflow.com/questions/2193307/how-do-i-get-latex-to-hyphenate-a-word-that-contains-a-dash
% в текущем документе работает без подключения дополнительных модулей (в смысле, необходимые модули уже и так подключены выше)
% заменяю только в тех случаях, где после экспорта в pdf строки, действительно, вылезают за границы
% (в консоли вывода pdflatex в этом случае будет предупреждение "Overfull \hbox")

% Если после двоийных кавычек идет пробел, при генерации pdf пробел уничтожается:
% например, в исходнике 'myth," he', превратится в pdf в 'myth,"he'
% В тексте использую кавычки ёлочки «»,
% если кавычки внутри кавычек (в цитатах), двойные обычные кавычки заменяю на лапки „“ или другие верхние двойные кавычки “”
% (лапки с нижней кавычкой нужно использовать в русских текстах, а две верхние кавычки - в английских)
% https://habr.com/ru/articles/865866/

% \raggedright внутри \section - отключить перенос слов внутри заголовка
% https://www.linux.org.ru/forum/general/11676660

\usepackage{graphicx}

\title{Экономические эффекты цифровизации в масштабах мировой экономики}
\author{А.~Е.~Моисеев}
\date{май 2024}
%\date \today

\graphicspath{ {img/} }

\begin{document}

\maketitle

\begin{center}
\textit{на правах рукописи}
\end{center}

% версии
% первая финальная: Статья-часть3 - Экономика хайтека - масштаб общества - 12.05.2024-final.odt
%   - май 2025
%   - весь основной текст и ссылки
% вторая финальная: Статья-часть3 - Экономика хайтека - масштаб общества - 09.08.2024-final-fix.odt
%   - август 2025
%   - убрал абзац про стелс и HaaS (ушел в отдельную статью)
%   - это базовая версия перед редактурой перед публикацией


Общество направляет труд и богатство на разработку технологических инженерных проектов для того, чтобы в будущем по результатам внедрения новых разработок в реальную экономику создать еще больше богатства или сэкономить еще больше труда \cite{ecoEffects}. Ресурсы, потраченные на разработку, компенсируются позднее — из экономического эффекта от внедрения инженерных разработок в реальную экономику \cite{businessModels}. Часть этих новых средств может быть вновь инвестирована в создание новых технологий, которые, будучи внедрены, в еще большей степени повысят производительность труда, цикл повторится. На каждом витке цикла экономика общества восходит по ступеням все более и более высокой производительности.

Таким образом, для того, чтобы разработка прошла полный цикл, в обществе должны быть в наличии те или иные реализации трёх ключевых механизмов. При наличии свободных ресурсов — накопленного капитала в количестве, достаточном, чтобы профинансировать реализацию технологического продукта, — при наличии команд квалифицированных специалистов, способных реализовать продукт, должен существовать действующий механизм доступа команд к капиталу. Механизм доставки разработанного продукта потребителям — предприятиям и организациям различных секторов экономики, — для действительной реализации экономического эффекта. Механизм компенсации средств, потраченных на разработку технологического продукта, — бизнес"=модели продаж инженерных продуктов, — обратное распределение экономического эффекта в сторону технологических компаний для компенсации первоначальных издержек на старые разработки и аккумуляции средств на финансирование новых проектов.

Особенность технологических инженерных проектов в области разработки ПО заключается в том, что цикл разработки и внедрения технологических продуктов никогда в истории не был таким недорогим и коротким, как с технологиями, основанными на цифровых вычислениях.

На современном этапе развития мировой экономики действующие реализации этих механизмов принимают множество форм, иногда переплетающихся, иногда перетекающих одна в другую. В условиях мирового разделения труда и мировой глобальной экономики разработчики и внедряющая сторона чаще всего — разные организации, разделенные юридически и географически внутри одной страны или распределены по множеству стран по всему миру. Накопленный капитал и команды специалистов концентрируются в мировых центрах технологических разработок\footnote{«В США сосредоточена почти половина глобального рынка услуг, связанных с программированием. Доля рынка, которая приходится на Китай, меньше более чем в 20 раз.» \cite{disserNagorniDigitalTransformWorld2021}}. Рынок внедрения технологических продуктов и действительной реализации экономических эффектов, обслуживаемый технологическими гигантами, охватывает весь мир.

Чтобы увидеть базовые механизмы влияния технологий цифровизации на экономику в целом, при этом не отвлекаясь на частные проявления отношений между игроками на региональных и мировых рынках, возьмем к рассмотрению всю мировую экономику целиком как единую мировую фабрику.

\section*{\raggedright{Экономические эффекты цифровизации в масштабах общества: мировая фабрика}}

В масштабах мировой фабрики производства материальных благ все производительные рабочие вместе взятые производят весь вместе взятый материальный продукт — продукты потребления, станки, энергию, сырые материалы, объекты недвижимости, прочие продукты. Часть этих благ составит жизненные средства самих производительных рабочих. Другая часть — ресурсы и продукты, — будет направлена на компенсацию необходимых непроизводственных издержек на управление и организацию производственного процесса. В масштабах мировой фабрики вместе с незанятыми непосредственно в производстве сотрудниками коммерческих предприятий в этот сегмент войдут в том числе расходы на чиновников и государственный аппарат, а также весь банковский и финансовый сектор. Еще часть составит жизненные средства и материальные условия труда работников сферы услуг — образование, медицина, красота, пассажирские перевозки, развлечения и т.~п. Еще одна часть составит жизненные средства людей, по тем или иным причинам не принимающих участие в системе производства: неработающие пенсионеры, нетрудоспособные инвалиды, неработающие студенты, безработные, живущие на пособие по безработице или средства т.~н. безусловного базового дохода, и т.~п. Еще одна часть будет направлена на создание продуктов интеллектуальной сферы — научные исследования и инженерные проекты, в том числе разработка программного обеспечения. И еще одна часть — после вычета перечисленных долей, — доход «коллективного предпринимателя».

Научные и инженерные разработки, будучи внедрены в тот или иной сегмент экономики, повышают производительность труда в целевом сегменте, создают экономические эффекты, открывают возможные пути для их реализации движением ресурсов и рабочей силы внутри экономики между теми или иными сегментами мировой фабрики.

Если в масштабах предприятия реализация экономического эффекта от повышения производительности при внедрении технологического продукта может быть конвертирована в деньги через прямое сокращение рабочей силы, то в масштабах общества следует учитывать, что сокращенные работники просто так никуда не исчезают — они остаются внутри исходного сегмента, перейдя на другое предприятие, или переходят в другой сегмент, полностью меняя вид деятельности. В том случае, если рассматриваемая технология внедрена не достаточно широко, такого рода движения остаются по большей части незаметны, т.~к. малые движения рабочей силы компенсируются неравномерностью развития производства и рынка труда. Если технология в значительной степени повышает производительность целевой профессии и внедрена достаточно широко, её влияние становится заметным в масштабах региональной или мировой экономики.

В том случае, если технология повышает производительность труда в сегменте материального производства, её внедрение может быть реализовано по сценарию увеличения выпуска продукта при сохранении старых затрат труда\footnote{А. Эйнштейн: «Изобрести — это значит увеличить числитель в следующей дроби „произведенные товары“/„затраченный труд“» \cite{einsteinInventor1982}} (Рис. \ref{fig:pic1_eco_effect_more_product}). Экономический эффект в таком случае составит добавочный материальный продукт — новое богатство, которое может увеличить потребление общества в пределах существующей численности или стать основой для расширения экономики и общества — роста населения. Другой вариант внедрения той же технологии может предполагать сохранение текущего объема производства с сокращением занятой в производстве рабочей силы (Рис. \ref{fig:pic2_eco_effect_same_product}). Экономический эффект для владельцев отдельных предприятий, внедривших технологию, в таком случае составит сокращение затрат на покупку рабочей силы. В масштабах мировой фабрики лишняя рабочая сила при этом переместится в сегмент незанятых до тех пор, пока в текущем или других сегментах на нее не будет создан дополнительный спрос. Еще один сценарий внедрения — сохранение текущего объема производства и текущего количества занятых в производстве рабочих с сокращением рабочего дня (Рис. \ref{fig:pic3_eco_effect_less_worktime}).

\begin{figure}[ht]
    \centering
    \includegraphics[width=0.95\textwidth]{pic1_eco-effect-more-product}
    \caption{Больше богатства прежними силами}
    \label{fig:pic1_eco_effect_more_product}
\end{figure}

\begin{figure}[ht]
    \centering
    \includegraphics[width=0.95\textwidth]{pic2_eco-effect-same-product}
    \caption{То же богатство меньшими силами}
    \label{fig:pic2_eco_effect_same_product}
\end{figure}

\begin{figure}[ht]
    \centering
    \includegraphics[width=0.95\textwidth]{pic3_eco-effect-less-worktime}
    \caption{Реализация экономического эффекта сокращением рабочего дня}
    \label{fig:pic3_eco_effect_less_worktime}
\end{figure}

В том случае, если технология повышает производительность труда в сегменте необходимых непроизводственных издержек, экономический эффект составит экономию на рабочей силе этого сегмента, сокращенная рабочая сила переместится в сегмент незанятых (Рис. \ref{fig:pic4_eco_effect_less_expenses}). Непроизводственные издержки не создают новый продукт, но обеспечивают нормальное течение производственного процесса. В масштабах организации, специализирующейся на обслуживании других организаций, выполняя для них необходимые непроизводственные задачи, к примеру, ведение бухгалтерского учета, оказание юридических услуг или обеспечение сбыта через предоставление рекламной площадки, повышение производительности может привести к тому, что один и тот же сотрудник будет выполнять большее количество старых задач. Однако в масштабах мировой фабрики это же повышение приведет к сокращению сектора в целом, т.~к. при неизменных размерах всех прочих секторов количество необходимых непроизводственных задач останется прежним, но их будет выполнять меньшее количество человек. Так же, как в сегменте производства, повышение производительности в сегменте непроизводственных издержек может быть реализовано сокращением рабочего дня работников этого сегмента.

\begin{figure}[ht]
    \centering
    \includegraphics[width=0.95\textwidth]{pic4_eco-effect-less-expenses}
    \caption{То же богатство меньшими издержками}
    \label{fig:pic4_eco_effect_less_expenses}
\end{figure}

Экономический эффект, полученный на текущем цикле внедрения, может быть направлен в любой из представленных сегментов или распределен между ними в некоторой пропорции. Например, за счет сокращения рабочей силы на производстве (Рис. \ref{fig:pic5_eco_effect_manufactura2service}) или в сегменте необходимых непроизводственных издержек (Рис. \ref{fig:pic6_eco_effect_nonmanufactura2service}) можно расширить сектор услуг, направив ресурсы и освободившуюся рабочую силу на новые рабочие места в этот сегмент. Или направить добавочное богатство на расширение сектора исследований и разработок с таким расчетом, чтобы на новом цикле разработки получить еще больше технологических продуктов и на очередном цикле внедрения получить еще больший экономический эффект. Решение, куда направить полученные реализацией эффекта ресурсы и средства, принимает распорядитель, уполномоченный субъектом экономики, в чьих руках был реализован экономический эффект, а также технологическая компания"=разработчик внедренного продукта в рамках той доли реализованного экономического эффекта, которую она получает через свою бизнес"=модель.

\begin{figure}[ht]
    \centering
    \includegraphics[width=0.95\textwidth]{pic5_eco-effect-manufactura2service}
    \caption{Перераспределение экономического эффекта, реализованного в секторе производства, в сектор услуг}
    \label{fig:pic5_eco_effect_manufactura2service}
\end{figure}

\begin{figure}[ht]
    \centering
    \includegraphics[width=0.95\textwidth]{pic6_eco-effect-nonmanufactura2service}
    \caption{Перераспределение экономического эффекта, реализованного в секторе непроизводственных издержек, в сектор услуг}
    \label{fig:pic6_eco_effect_nonmanufactura2service}
\end{figure}

В действительности механизм перераспределения высвобожденных ресурсов и рабочей силы между сегментами может быть реализован разными путями. Например, экономический эффект может быть сначала превращен в доход предпринимателя, далее этот добавочный доход может быть использован для создания частного образовательного учреждения или частной клиники, т.~е. направлен в сегмент услуг. В другой ситуации частью добавочного продукта от реализации экономического эффекта распорядится государство, получившее дополнительные налоги или дополнительный доход госкорпораций. Государство может направить добавочный доход для создания государственных школ, спортивных комплексов, больниц, таким образом реализованный экономический эффект снова попадет в сегмент услуг, но по другому пути. Новые рабочие вакансии заполнит высвободившаяся рабочая сила. Спрос на дополнительные услуги обеспечат предприниматели и оплатят его частью добавочного дохода, оставшейся в личном потреблении, или все прочие члены общества, которые смогут оплатить новые услуги в том случае, если часть экономического эффекта была реализована повышением зарплаты, или же услуги оплатит государство через фонд оплаты труда работников бюджетных учреждений.

Если из сегмента в сегмент двинется рабочая сила, но не двинется материальный ресурс, освобожденный реализацией экономического эффекта, то внутри сегмента возрастет конкуренция между специалистами, среднее потребление упадет. Для того, чтобы среднее потребление в целом росло, часть технологий должна быть нацелена на сектор материального производства, реализация экономических эффектов в котором отчасти должна происходить по сценарию создания добавочного продукта при более высокой производительности.

\section*{\raggedright{Направление ресурсов в сегмент технологий}}

Для инициации новых проектов в области исследований и разработок следует выделить материальный ресурс — здания, оборудование, энергию, жизненные средства исследователей и инженеров, — они будут направлены в сегмент исследований и разработок. А также привлечь специалистов на созданные рабочие позиции в сегменте. Материальный ресурс выделяет инвестор — распорядитель богатства, имеющий возможность направить на реализацию проекта накопленный капитал в достаточном количестве. Источником первоначального капитала может послужить, к примеру, сегмент, представляющий доход предпринимателей. Часть производительного труда, которая на предыдущих экономических циклах создавала материальный продукт, потреблявшийся как доход, на новом цикле инвестиций будет использована для создания необходимой инфраструктуры для разработки и жизненных средств разработчиков (Рис. \ref{fig:pic7_invest2tech}).

\begin{figure}[ht]
    \centering
    \includegraphics[width=0.95\textwidth]{pic7_invest2tech}
    \caption{Первоначальные инвестиции в сегмент технологий}
    \label{fig:pic7_invest2tech}
\end{figure}

В том случае, если для нового проекта будет вновь подготовлена необходимая материальная база, а на новые рабочие позиции будут привлечены специалисты из существующих технологических компаний, произойдет движение рабочей силы внутри сегмента — сегмент расширится по количеству выделенных ресурсов, но не по количеству квалифицированной рабочей силы. В такой ситуации возникнет конкуренция за специалистов между организациями, имеющими в них потребность, потребление внутри сегмента, в целом, вырастет, но возможность реализовать дополнительную массу проектов в этом случае не появится. Для того, чтобы сегмент технологий рос также по количеству задействованных специалистов, которые смогут разработать больше технологических продуктов, имеющих шансы на внедрение, требуется привлекать работников из других сегментов. К примеру, из сегмента незанятых — студентов соответствующих специальностей, завершающих цикл обучения. Из прочих сегментов — материального производства, непроизводственных издержек, услуг, — специалистов, имеющих опыт в своей профессии, которые решили перейти в интеллектуальную сферу деятельности, при необходимости пройдя переобучение на краткосрочных курсах или полный цикл обучения новой профессии. Поводом и стимулом для такого перемещения может послужить предварительное заметное повышение производительности труда в том или ином сегменте, которое было реализовано сокращением рабочей силы.

Направление ресурсов в сегмент технологий также вызовет потребность в некотором расширении сегмента непроизводственных издержек — руководство, менеджмент, учет для вновь созданных технологических компаний, — а также сегмента услуг — подготовка дополнительной массы необходимых специалистов.

Сокращение сектора исследований и разработок происходит по аналогичному механизму, но в обратном направлении: часть материальных ресурсов, поступающих из источника, который на текущем цикле разработок был направлен на разработчиков и создание необходимых условий их труда, на новом цикле направляются не на создание условий для реализации нового или продолжения старого незавершенного проекта, а превращаются в доход\footnote{«The main reasons for recent layoffs are gloomier economic expectations (many companies and analysts expect a recession this year) and a desire to deliver some value for shareholders after a very tough 2022 for stock prices. If that value can’t come from the top line in the form of revenue, it has to come from the bottom line by cutting expenses, like payrolls.» \cite{businesscomBigProfitsBigLayoffs2013}}. Специалисты сегмента технологий при этом отправляются в сегмент незанятых, оказывая давление на рынке труда на пока ещё не сокращённых специалистов, или находят новое занятие в других сегментах. Сокращение сегмента исследований и разработок не нарушит циклический процесс воспроизводства материальных благ и необходимых услуг, они продолжат работу с уже внедренными ранее технологиями с текущей производительностью. Однако рост производительности труда, обеспечиваемый внедрением новых технологий, при этом прекратится, прекратится рост экономики, обеспечиваемый реализацией экономических эффектов регулярного повышения производительности труда.

Затраты, уже понесенные инвесторами на финансирование прерванных незавершённых разработок, будут списаны как убытки. В том случае, если они были велики, велик будет и ущерб — не доведенный до завершения продукт не имеет шансов на внедрение, значит расходы на него в масштабах общества никаким образом не смогут окупить первоначальные инвестиции. Отдельные игроки могут перепродавать друг другу доли в компаниях, создающих новый интеллектуальный продукт, и права на незавершённые проекты, компенсируя таким образом собственные инвестиции ранних этапов\footnote{«WeWork Cos. co-founder Adam Neumann has cashed out more than \$700 million from the company ahead of its initial public offering through a mix of stock sales and debt, people familiar with the matter said — an unusually large sum given that startup founders typically wait for the IPO to monetize their holdings.» \cite{wallstreetWeWorkCashed2019}}. Но в том случае, если проект в финале все равно будет остановлен или будет внедрен недостаточно широко, убыток понесет последний из покупателей-инвесторов\footnote{«WeWork filed for Chapter 11 bankruptcy and its stock fell to 84 cents a share, giving it a \$44.5 million valuation.» \cite{forbesWeWorkTimeline2023}}. Технологии разработки программного обеспечения такие, как аджайл (гибкая разработка) и mvp (минимальная полезная модель), ставящие во главу угла стремление минимизировать циклы разработки до этапа внедрения, обеспечивая «непрерывную» доставку результатов промежуточных этапов разработки в производственный процесс потребителя технологического продукта\footnote{«Deliver working software frequently, from a couple of weeks to a couple of months, with a preference for the shorter timescale.» \cite{agileSoftDevel2002}}, минимизируют потери от реализации риска прекращения разработки в произвольный момент времени, т.~к. в убыток в таком случае спишется только последний не дошедший до стадии внедрения этап разработки.

После того, как средства на новый проект выделены из сегмента дохода, начнется цикл разработки, который завершается созданием технологического продукта, внедрением его в производственный процесс в целевом сегменте, созданием экономического эффекта. После того, как интеллектуальный продукт разработан и готов к внедрению, владелец продукта больше не испытывает потребность в команде специалистов, по крайней мере в таком размере, в котором в ней была потребность на этапе разработки. Источник средств инвестора, который он использовал для создания материальных условий разработки, может быть снова направлен для создания личного дохода или инвестиций другого рода в другие сегменты, в этом случае повторится часть сценария сокращения сегмента разработок. Т.~к. технологический продукт разработан и внедрен, в целевом сегменте внедрения возникнет «освободившаяся» дополнительная масса ресурсов — экономический эффект. Эти ресурсы полностью или частично могут быть направлены вновь в сегмент технологий для реализации нового проекта, т.~е. реинвестированы в разработки (Рис. \ref{fig:pic8_reinvest2tech}). В том случае, если суммарный экономический эффект превысит исходную стоимость разработки и его реинвестированная в технологии часть тоже будет равна или превысит её стоимость, сегмент разработок не уменьшится, а останется в прежнем размере или вырастет.

\begin{figure}[ht]
    \centering
    \includegraphics[width=0.95\textwidth]{pic8_reinvest2tech}
    \caption{Повторное инвестирование в сегмент технологий}
    \label{fig:pic8_reinvest2tech}
\end{figure}

Механизм реинвестирования на практике может быть реализован по разным сценариям. К примеру, если разработка велась внутренней командой инженеров производственного предприятия и была успешно внедрена в производственный процесс компании, руководство предприятия может решить направить часть добавочной прибыли от реализованного экономического эффекта на финансирование нового проекта силами этой же команды. Если технологическая компания продает бессрочные лицензии на «чистое» программное обеспечение, потребитель перечисляет разработчику цену лицензии, которую позднее компенсирует из части реализованного экономического эффекта. Разработчик программного продукта собирает плату со всех потребителей на рынке, агрегированная сумма может быть реинвестирована самим разработчиком в разработку новой версии продукта или создание нового продукта, который он, имея достаточно широкую базу клиентов и авторитетный бренд, сможет продать на новом цикле производства с достаточно высокими шансами внедрить его достаточно широко. Специалисты"=разработчики внутри технологической компании в этом случае могут вообще не заметить промежуточную стадию перерыва в работе между циклами первичного инвестирования и реинвестирования, т.~к. после завершения работы над первым продуктом, компания сразу перенаправит их на разработку следующего продукта, компания будет иметь относительно постоянный штат специалистов, процесс разработки будет выглядеть непрерывным также и со стороны. В том случае, если рынок потребителей лицензионного продукта исчерпан, денежный поток со стороны потребителей ПО иссяк, а компания не смогла разработать новый успешный продукт или не захотела реинвестировать накопленные средства, ее деятельность может прекратиться, произойдет сокращение сегмента технологий. Однако, в том случае, если продукт, не требующий новых лицензионных отчислений, или изначально доступный бесплатно, к примеру, под свободной лицензией, продолжает функционировать, он будет продолжать создавать экономический эффект относительно ситуации, когда он не был внедрен, т.~е. оставлять добавочные средства на стороне внедрившего его потребителя. Эти средства каждый из потребителей может самостоятельно направить в сегмент технологий — к примеру, нанять во внутренний штат разработчиков программного обеспечения. В таком случае реализация экономического эффекта от старой разработки продолжит питать сегмент разработок в целом, сжатия сегмента по количеству ресурсов и специалистов после закрытия исходной технологической компании в окончательном итоге может не произойти. Однако, следует заметить, что разработки крупных технологических компаний внедряются максимально широко, малая цена единичной лицензии при большом количестве покупателей сливается в заметную сумму в распоряжении единой организации, которая может в таком случае позволить себе разработать достаточно сложный новый программный продукт. Каждый из множества потребителей, на стороне которого реализуется экономический эффект внедрения старого программного продукта, получает в распоряжение лишь малую часть суммарного общественного экономического эффекта, его не хватит на разработку нового продукта такой сложности, который может себе позволить разработать крупная технологическая компания. Кроме того, масштаб внедрения продуктов программистов из внутреннего штата компании нетехнологического сектора обычно ограничен размерами самой компании, значит ограничен максимальный экономический эффект от их разработок — он будет меньше экономического эффекта от внедрения аналогичного продукта крупной технологической компании, рынком сбыта которой является весь мир.

В том случае, если процесс реинвестирования реализуется циклично, сегмент разработок на каждом очередном цикле создаёт все новые интеллектуальные продукты, черпает из прочих сегментов экономики все новые экономические эффекты, которые вновь инвестируются в создание новых технологий. В этом случае создаётся видимость, что технологии более не нуждаются в первоначальных инвестициях из нетехнологических сегментов, т.~к. часть реализованного эффекта от старой технологии создаёт условия создания новой технологии, в рамках этой логики сегмент технологий обеспечивает и воспроизводит сам себя.

В действительности, текущая деятельность технологической компании и регулярное перераспределение части реализованного экономического эффекта в ее пользу становится необходимым условием сохранения новой повышенной производительности на достигнутом уровне у ее клиентов. В том случае, если внедренный технологический продукт будет изъят из установившихся рабочих процессов, производительность понизится до старого уровня, экономический эффект проявится со знаком минус. В программном продукте, продававшемся по модели бессрочной лицензии компанией, прекратившей деятельность или прекратившей поддержку старого продукта, может обнаружиться критическая проблема, которая сделает невозможным его дальнейшую эксплуатацию без исправлений, которые в отсутствие разработчика никогда не будут сделаны. Технологическая компания, продающая «чистое» ПО по модели подписки, может отозвать право на использование ПО\footnote{«Autodesk, американский разработчик систем автоматизированного проектирования (САПР) для строительства, проектирования и промышленного дизайна, с 20 марта [2024] запретила использование своих программных продуктов российскими компаниями. Об этом говорится в письме, которое Autodesk направила своим партнерам в России.» \cite{rbcAutodeskForbidsRussia2024}}. Регулярная покупка доступа к аппаратному обеспечению технологических компаний, продающих облачные сервисы, для покупателя сервиса становится необходимым условием ведения текущей деятельности с установившейся повышенной производительностью, поэтому прекращение доступа к сервису\footnote{«Американская компания Microsoft отправила российским компаниям письмо о том, что c 20 марта 2024 года закроет доступ к своим облачным продуктам для российских организаций.» \cite{rbcMicrosoftCancelCloudRussia2024}} в отсутствие сопоставимых альтернатив приведет к понижению производительности на стороне бывшего потребителя\footnote{первый вице"=президент Газпромбанка: «Если мы немедленно проведем импортозамещение, бизнес"=процессы встанут» \cite{tadviserGazprombankSubstImport2024}}, т.~е. в конечном итоге приведет к излишним тратам или прямой потере средств.

Эффект от технологических разработок является накапливаемым: если разработка не окупила затраты на текущем экономическом цикле внедрения, но ее использование было продолжено, эффект будет проявлять себя и накапливаться на следующих циклах и окупит первоначальные расходы в более долгой перспективе. Фактически, эффект проявляет себя все время, когда технология применяется в производстве, даже если денежный поток не идёт от потребителя технологии к разработчику, выгоду получает потребитель технологии, в масштабах мировой фабрики — всё общество. От инженерных разработок ожидают относительно короткого срока окупаемости. В случае с научными разработками нормальным является обстоятельство, что их эффект влияния на экономику обычно проявляет себя в удаленной перспективе. При том, что на каждом цикле общество продолжит нести расходы на сегмент технологий, они расходуются на создание новых разработок, которые проявят себя дополнительными экономическими эффектами, не имеющими прямого отношения к рассматриваемому текущему циклу внедрения. Реализация экономического эффекта от внедрения разработок в любом случае удалена во времени от цикла, в рамках которого они проведены\footnote{Microsoft: «Significant revenue from new product and service investments may not be achieved for a number of years, if at all.» \cite{microsoft10K2006}}.

Таким образом, размер экономического эффекта не привязан к стоимости разработки. Фактически он ограничен только размером той части экономики, в рамках которой может быть применена технология, и временем применения технологии с момента её внедрения. Разработчики организуют обратный денежный поток со стороны внедряющих их технологии и реализующих действительный экономический эффект предприятий в свою сторону при помощи тех или иных бизнес"=моделей.

В 1999-2001 годы компания Майкрософт разрабатывала операционную систему Windows XP \cite{winsupersiteWindowsXP2001} и некоторые другие крупные программные продукты такие, как офисный пакет MS Office XP, платформа .Net для разработчиков программного обеспечения. Расходы на исследования и разработки за этот период в общей сложности составили \$11 млрд.\footnote{Отчет Microsoft 10-K за 2001, «Research and development» (в млн.): 2,970 (1999) + 3,772 (2000) + 4,379 (2001) = \$11,121 (\$11 млрд.)» \cite{microsoft10K2001}}, часть этой суммы была потрачена на разработку Windows XP. Продажи Windows XP начались в октябре 2001, в январе 2007 года была выпущена Windows Vista. Значительную долю выручки по статье «настольные операционные системы» (в отчетах — «Client»\footnote{«Client segment includes Windows XP, Windows 2000, and other standard Windows operating systems.» \cite{microsoft10K2003}}) за этот период времени с 2001 по конец 2006 года составили продажи лицензий на Windows XP, вся выручка по этой статье за это время составила \$56.7 млрд.\footnote{Отчеты Microsoft 10-K за 2003, 2006, «Revenue: Client» (в млн.): 9,360 (2002) + 10,394 (2003) + 11,556 (2004) + 12,151 (2005) + 13,209 (2006) = \$56,670 (\$56.7 млрд.) \cite{microsoft10K2003, microsoft10K2006}} и, таким образом, окупила все разаботки предшествующего периода, включая разработку самой Windows XP. Компания Майкрософт ведет разработку преимущественно в США\footnote{Microsoft: «Our research and development facilities are located primarily in Redmond, Washington. We also have smaller research facilities in other parts of the United States and around the world, including, but not limited to, China, Denmark, England, India, Ireland, and Israel. <...> As of June 30, 2006, we employed approximately 71,000 people on a full-time basis, 44,000 in the United States and 27,000 internationally.» \cite{microsoft10K2006}}, продажа разработанных продуктов, а значит внедрение и реализация экономических эффектов, происходят в множестве стран по всему миру\footnote{Microsoft: «We operate in over 100 countries and a significant part of our revenue comes from international sales.» \cite{microsoft10K2006}}. В течение 2006-2023 Майкрософт ежегодно реинвестировала в разработки 12-15\% выручки, с 1998 по 2005 — 14-22\% \cite{microsoftSecProfile}. В 2000 году в отделе исследований и разработок компании трудилось 16 тыс. сотрудников, в 2023 — 72 тыс. человек \cite{microsoftSecProfile}.

К действительным технологическим компаниям следует относить такие компании, которые, получая доход от продажи технологических продуктов на рынке, реинвестируют значительную долю дохода в исследования и разработки новых технологических продуктов, а также регулярно выводят на рынок новые технологические продукты, которые, будучи внедрены в целевых сегментах мировой фабрики, ещё больше повышают общемировую производительность труда. В том случае, если компания, продающая на рынке программное обеспечение по модели подписки, перестает выводить на рынок новые продукты или если выводимые на рынок улучшения старого успешного продукта не создают в точках внедрения достаточного добавочного экономического эффекта, можно говорить о том, что такая компания из центра технологических разработок и источника повышения производительности мировой фабрики выродилась в своеобразного сборщика «налога» на программное обеспечение\footnote{«U.S. regulators sued Adobe on Monday over claims that the company made it difficult to cancel subscriptions to Photoshop and other software, an escalation by regulators in a crackdown against such practices. <...> In recent years, Adobe has shifted to offering subscriptions for those products, scrapping its previous model of selling one-off licenses to use the tools. <...> The company earned \$14.22 billion in revenue from subscriptions in 2023, up from \$7.71 billion in 2019, the government said. <...> Adobe took steps to lock consumers into yearly subscriptions billed in monthly increments, the lawsuit argued. <...> “Some of these subscribers do not realize for months that Adobe is continuing to charge them, and only learn about the charges when they review their financial accounts.”» \cite{nyTimesUsSuesAdobe2024}}, питаемого эффектом повышенной производительности, достигнутой на предыдущих экономических циклах. Технологические компании, продающие доступ к собственной облачной инфраструктуре, в действительности продают на рынке не программное обеспечение, а аппаратное обеспечение, затрачиваемую энергию и труд обслуживающего персонала (в т.~ч. системных администраторов и контент"=модераторов). Регулярное обновление аппаратного обеспечения\footnote{«Microsoft, Meta, and Google’s parent company, Alphabet, disclosed this week that they had spent more than \$32 billion combined on data centers and other capital expenses in just the first three months of the [2024] year.» \cite{nyTimesInRaceToBuildAi2024}} и регулярный труд обслуживающего персонала становится условием продолжения присутствия облачного сервиса на рынке, регулярная покупка доступа к этой инфраструктуре и ее потребление в рамках текущей деятельности — условием сохранения установившейся повышенной производительности труда на стороне клиентов сервиса. В том случае, если такая компания перестает реинвестировать доход от продажи облачной инфраструктуры в создание новых продуктов и значительные улучшения существующих продуктов, можно говорить о том, что она из технологической компании превращается в компанию, ведущую деятельность преимущественно в специфической части сегмента материального производства.

\section*{\raggedright{Фонд свободного времени}}

Оплата труда специалиста на основном месте работы включает необходимые жизненные средства, которые обеспечивают жизнедеятельность человека не только во время его присутствия на рабочем месте, но и все остальное время, которое для него выступает как свободное. Кроме выполнения необходимых личных дел — перемещение по городу, домашний быт, — свободное время может быть потрачено на отдых, общение, спорт, творчество, потребление культурного контента, самостоятельное обучение, обучение в образовательных организациях, потребление дополнительной массы услуг и т.~п. Кроме того, свободное время может быть использовано для интеллектуального труда — инженерно"=технического творчества, в том числе для разработки программного обеспечения. Т.~к. специалист, имеющий досуг, распоряжается этим временем по своему усмотрению, а необходимые жизненные средства он получает за работу на основном месте (причем, в любом сегменте деятельности, не обязательно в сегменте исследований и разработок, — на производстве, в офисе, в сфере услуг), плоды интеллектуального труда, созданные в свободное время, не требуют от общества привлечения дополнительных ресурсов — они появляются фактически «бесплатно». Однако, кроме этого они фундаментально более ничем не отличаются от разработок, созданных специалистами"=инженерами в основное рабочее время, т.~е. выступающих как плоды оплаченного интеллектуального труда.

Интеллектуальные продукты, созданные во время досуга по личной инициативе разработчика, могут быть внедрены в экономику и создать реальный экономический эффект наряду с интеллектуальными продуктами, созданными профессионалами в основное оплаченное рабочее время\footnote{Microsoft: «The Internet as a distribution channel and the non"=commercial software model described above have reduced barriers to entry even further. Open source software vendors are devoting considerable efforts to developing software that mimics the features and functionality of our products.» \cite{microsoft10K2006}}. Современные технологии разработки программного обеспечения позволяют вести совместную разработку группам специалистов, распределенных по миру во времени и пространстве. Каналы доставки через интернет позволяют внедрять такие продукты любому количеству потребителей в любой точке мира. Многие программные продукты, имеющие открытую модель разработки и распространяемые с исходным кодом под свободной лицензией, включают как плоды труда инженеров, ведущих разработку в свободное время досуга, так и инженеров, получающих зарплату за вклад в этот же проект, вносимый в основное рабочее время по заданию работодателя.

Ядро операционной системы Линукс, создание которого начиналось как инициативный персональный проект студента, в настоящее время разрабатывается преимущественно корпоративными разработчиками. При разработке ядра Линукс версии 6.1, выпущенной в 2022 году, с момента её предыдущего релиза индивидуальные разработчики отправили всего 4\% наборов изменений \cite{lwnLinuxKernelDevelStats2022}, что составляет заметный, но не очень большой процент по сравнению с общим вкладом инженеров — сотрудников корпораций. Однако, с вкладом каждой отдельной корпорации общий вклад индивидуальных разработчиков оказывается соизмерим: наибольший вклад в эту же версию ядра внесла Huawei Technologies — 9.2\%, среди прочих разработчиков вклад Intel — 9\%, Red Hat — 4.8\%, Meta (запрещена в РФ) — 3\%, Nvidia — 2.8\%. Индивидуальные (некорпоративные) разработчики по этому критерию оказываются на 8-м месте, если не принимать в рассмотрение вклад разработчиков, у которых статус принадлежности или непринадлежности к корпорации в исследовании не установлен (помечен как «Unknown»). Если же предположить, что не идентифицированные разработчики являются индивидуалами, общий вклад инициативных разработчиков окажется на первом месте с 11.9\%, превышая вклад каждой отдельной корпорации.

В разработку проекта по созданию графического окружения рабочего стола KDE, одного из крупнейших проектов с открытым исходным кодом, ежегодно вносит вклад около тысячи человек, на более"=менее постоянной основе с 2008 года — 600-800 человек \cite{carlschwanKdeCommunity2023}. Проект начинался как индивидуальная разработка, потом к ней присоединились единомышленники, которые вносили улучшения тоже по личной инициативе\footnote{«In 1996, Matthias Ettrich found himself frustrated with the absence of a user-friendly, inexpensive work environment for the Linux operating system. <...> Matthias had the idea and the basic framework for an improved alternative in mind, but he knew he’d need help to create the platform. So he posted a call to action asking the Linux development community to join him, and, thus, KDE was born.» \cite{hostingadviceKdeCommunity2017}}. Наиболее влиятельным в современной индустрии развитием проекта оказался движок для отображения веб-страниц WebKit, за основу которого был взят внутренний модуль проекта KDE — KHTML. В настоящее время WebKit разрабатывается за пределами проекта KDE, используется внутри веб-браузера Google Chrome и множества браузеров, созданных на его основе, является абсолютным монополистом среди инструментов отображения содержимого веб-страниц. Несколько корпоративных спонсоров поддерживают финансово некоммерческую организацию KDE e.V., однако ее деятельность преимущественно направлена на обеспечение информационной поддержки проекта, поддержание серверной инфраструктуры, организацию мероприятий (конференции, код-спринты), а не на оплату труда инженеров. О корпоративных разработчиках, вносящих вклад в проект в оплаченное рабочее время, известно меньше. В 2012-2013 годы компания Blue Systems наняла около 10 инженеров для работы над KDE (ссылки на соответствующие сообщения в личных блогах разработчиков собраны на странице в Википедии: https://en.wikipedia.org/wiki/Blue\_Systems). О 45-ти сотрудниках, ведущих работу над KDE, в 2012 году сообщала компания KDUB\footnote{«KDAB and its partners have been contracted by various customers to supply KDE development services. All of this code has made it back into KDE's codebase, and has helped to make KDE what it is today. <...> We recently signed Fiduciary License Agreements for all of our about 45 developers.» \cite{kdenewsKdabLicense2012}}. В 2022 году в компании Valve работал относительно большой штат программистов в направлении по участию в проектах с открытым исходным кодом\footnote{«Griffais says the company is also directly paying more than 100 open-source developers to work on the Proton compatibility layer, the Mesa graphics driver, and Vulkan, among other tasks like Steam for Linux and Chromebooks.» \cite{thevergeValveSteamDeck2022}}. Некоторые из них работают над проектом KDE\footnote{«Longtime open-source developer Roman Gilg is now working under contract for Valve. He will be focusing on “certain gaming-related XServer projects and improve KWin in this regard and for general desktop usage.”» \cite{phoronixValveIsFundingKde2019}}, т.~к. он выбран в качестве интерфейса рабочего стола продукта компании — игровой консоли Steam Deck. В 2022 году KDE e.V. сообщила, что впервые создала две вакансии, связанные с разработкой, — документация и работа с пакетами\footnote{«In 2022 KDE e.V. started hiring engineers for technical positions, beginning with a packaging engineer and a documentation specialist.» \cite{kdeevReport2022}}. Таким образом, хотя в проект вносят вклад инженеры в оплачиваемое рабочее время, есть основания предполагать, что проект KDE развивают преимущественно инициативные индивидуальные разработчики, в этой части своей деятельности не связанные с корпорациями.

По данным опроса участников проектов с открытым исходным кодом, проведенным GitHub в 2017 году, 65\% участников опроса публиковало код по заданию на основном месте работы \cite{githubOpenSourseSurvey2017}. Таким образом, 35\% участников этого опроса разрабатывают и публикуют код в собственное свободное время исключительно по собственной инициативе.

Среди участников опроса представителей российского рынка разработки ПО, проведенного изданием «N + 1» в 2023 году, 15\% разработчиков вносит вклад в проекты с открытым исходным кодом от лица компании, 68.9\% — по собственной инициативе, остальные не являются разработчиками \cite{nplus1OpensourseRussia2024}. Таким образом, среди опрошенных разработчиков, отправляющих код в свободные проекты, 18\% вносит вклад по заданию работодателя, 82\% — по личной инициативе.

Определенную известность в технологическом сообществе получило т.~н. «правило 20\%» из правил внутреннего распорядка компании Google, в рамках которого предполагалось, что разработчики компании имеют возможность тратить часть своего рабочего времени на собственные инициативные разработки. Однако, исходная формулировка «правила» подразумевала возможность вести работу не над личными проектами, а инициировать внутрикорпоративные разработки в относительно свободном режиме\footnote{«Founders Larry Page and Sergey Brin highlighted the idea in their 2004 IPO letter: “We encourage our employees, in addition to their regular projects, to spend 20\% of their time working on what they think will most benefit Google,” they wrote. <...> Huge 20\% products include the development Google News, Gmail, and even AdSense.» \cite{businessinsiderGoogle20percentTruth2015}}. Действие правила внутри компании в настоящее время также ставится под сомнение\footnote{«Google’s “20\% time,” which allows employees to take one day a week to work on side projects, effectively no longer exists. That’s according to former Google employees, one who spoke to Quartz on the condition of anonymity and others who have said it publicly. <...> Google began to require that engineers get approval from management to take 20\% time in order to work on independent projects, a marked departure from the company’s previous policy of making 20\% time a right of all Googlers. Recently, however, Google’s upper management has clamped down even further, by strongly discouraging managers from approving any 20\% projects at all.» \cite{qzGoogle20percentDead2013}}. 

Механизм появления коммерческих предприятий из персональных разработок, проведенных автором по личной инициативе, нашел отражение в культе т.~н. «гаражных» стартапов\footnote{«The garage is central to the origin of many corporate success stories in the twentieth century, from chauffeur to entrepreneur; the space originally intended for the storage of automobiles has become a symbol, a myth, a banal object in the domestic landscape that gave birth to the industrial tech complex.» \cite{fastcompanySilliconValleyGarageMyth2018}}. В рамках канона, первые шаги по реализации нового революционного продукта автор или небольшой коллектив реализуют в «личном гараже» \cite{dilbertGarage1}, используя при этом исключительно личное время и средства. После демонстрации успешности идеи разработка развивается в коммерческую корпорацию, привлекая для роста внешний капитал \cite{dilbertGarage2}. Хотя модель имеет под собой исторические корни и многие состоявшиеся продукты, действительно, начинались по личной инициативе без привлечения внешних ресурсов, под действием всеобщего положительного отношения к образу «гаражной» разработки состоявшиеся корпорации иногда включают такой этап в описания ранних этапов своей истории даже в том случае, если это не вполне соответствует действительности\footnote{«The development of the Google algorithms was carried on on a variety of Computers, mainly provided by the NSF-DARPA-NASA-funded Digital Library project at Stanford. Crawling the web to obtain its link structure required an enormous amount of storage in comparison with typical student projects at that time. We show here the original storage assembly, containing 10 4 Gigabyte disk drives, giving 40 Gbytes total.» \cite{stanfordOriginalGoogleComputer1996} (на ссылки указывает преподаватель МФТИ Юрий Аммосов https://t.me/kedr2earth/8591)}\footnote{«Wozniak, however, doesn't really see that location as a crucial part of Apple's history. “The garage is a bit of a myth,” he told Bloomberg Businessweek's Brandon Lisy when asked whether the garage was important to Apple's story. “We did no designs there, no breadboarding, no prototyping, no planning of products. We did no manufacturing there.”» \cite{businessinsiderAppleGarageMyth2014}}. Из этого можно заключить, что мифы такого рода, по их мнению, положительно влияют на общественный имидж компании, представляющей свою историю как инициативную «гаражную» разработку.

Так или иначе, общий интерес и, в целом, положительное отношение к «правилу 20\%», к «гаражным» стартапам, а также к модели открытой разработки свободного программного обеспечения в технологической среде отражают общее положительное отношение к действующим или предполагаемым «недирективным» механизмам инициации некоторых разработок, которые в дальнейшем, тем не менее, доходят до стадии внедрения в экономику. Люди, имеющие достаточные навыки и необходимое свободное время, могут быть склонны к тому, чтобы использовать его для инициативного научно"=технического творчества\footnote{«Many in the FLOSS world see themselves as <...> people who enjoy programming as a creative, problem-solving activity.» \cite{sociologySociologicalKDE2011}}.

Таким образом, часть фонда свободного времени, представленная досугом, может служить источником для появления и развития интеллектуальных технологических продуктов, которые не требуют выделения специальной доли общественного ресурса или принятия специальных управленческих решений со стороны структур, управляющих долей общественного богатства в частном порядке или от имени общества в целом. В том случае, если разработки такого рода будут реально внедрены в экономику общества, они проявят себя реализацией действительных добавочных экономических эффектов.

Основа для фонда свободного времени — высокая производительность труда в первую очередь в сегменте материального производства\footnote{«Экономия производительного труда используется для увеличения объема непроизводительного труда, свободного времени общества» \cite{zolotovManufacturingLabour2006}}, но также и в прочих сегментах. Фонд свободного времени пополняется через сокращение рабочего дня, которое может быть достигнуто реализацией экономического эффекта при очередном заметном повышении производительности труда в действующих сегментах мировой фабрики.

\section*{\raggedright{Пределы роста}}

Таким образом, технологии, в т.~ч. цифровое программное обеспечение, будучи внедрены, создают экономические эффекты в тех или иных сегментах экономической деятельности. Часть реализованного эффекта возвращается коммерческим технологическим компаниям в форме дохода, получаемого по той или иной бизнес"=модели. В том случае, если потенциал внедряемой технологии велик, ее распространение и внедрение у всех потенциальных потребителей займет некоторое время. В течение этого периода все новые потребители будут покупать и внедрять технологический продукт, компания"=разработчик будет получать стабильный и растущий доход. Рост будет продолжаться до тех пор, пока подавляющее большинство потребителей не внедрит у себя целевой продукт. После того, как это произойдет, денежный поток в сторону технологической компании"=разработчика в зависимости от выбранной бизнес"=модели иссякнет или сохранится на постоянном уровне, в любом случае перестанет расти. Общественная производительность труда, обеспечиваемая целевой технологией, достигла целевого уровня, ее рост прекратился. Можно говорить о том, что технология, будучи внедрена достаточно широко, исчерпывает себя. Для нового роста производительности требуется новая технология.

В масштабах мировой фабрики максимально широкое внедрение технологии — полное количество целевого автоматизируемого вида деятельности, представленное в своем целевом сегменте. Потенциал реализации общего экономического эффекта от всех вместе взятых технологий для некоторого целевого сегмента ограничен текущим размером целевого сегмента. Текущий размер целевого сегмента ограничит максимальный рынок для всех вместе взятых технологических компаний, представляющих продукты для автоматизации деятельности, относимой к этому сегменту.

\section*{\raggedright{Пределы роста: необходимые непроизводственные издержки}}

Количество труда, задействованного в сегменте непроизводственных издержек, определяется потребностями ведения текущей деятельности во всех прочих сегментах — материальном производстве, услугах, создании интеллектуального продукта. Таким образом, в нормальных условиях, когда потребности в специалистах, решающих текущие задачи такие, как управление, учёт, сбыт и т.~п., полностью покрыта, и деятельность ведётся в нормальном режиме, внедрение технологии автоматизации труда такого рода в масштабах мировой фабрики может привести только к сокращению сегмента необходимых непроизводственных издержек. Каждая новая технология, обеспечивающая заметное повышение производительности непроизводственного труда, будет приносить своему разработчику часть экономического эффекта — сэкономленный в сегменте ресурс. Сам сегмент при прочих неизменных условиях при этом будет сокращаться. Экономия ресурсов, достигнутая этим сокращением, и будет являться источником экономического эффекта, питающего сегмент технологий через новый доход технологических компаний. Таким образом, предел роста всех вместе взятых технологических компаний, создающих продукты для автоматизации труда преимущественно в непроизводственном сегменте, составит количество труда, которое можно потенциально сократить во всем вместе взятом сегменте непроизводственных издержек.

В том случае, если текущее поколение технологий внедрено в сегменте достаточно широко, или если производительность целевого непроизводительного труда достигла такой степени, что количество занятых в нем специалистов заметно сократилось в абсолюте, циклическая реализация эффекта от внедрений старых технологий всё ещё сможет обеспечить компаниям"=разработчикам постоянный доход, к примеру, через регулярные платежи за подписку на лицензию на программное обеспечение или на облачные сервисы. Но замедлившиеся темпы внедрения старых продуктов и расчетный абсолютный размер потенциального экономического эффекта от внедрения нового продукта не смогут обеспечить технологическим компаниям прежние темпы роста. Т.~к. в этой ситуации потенциал от внедрения новых эволюционных разработок окажется невелик, шансы окупить издержки на разработку станут также невелики, это приведет к тому, что сегмент технологий в части, нацеленной на сегмент непроизводственных издержек, также сократится. В том случае, если причина замедления роста в исчерпании старого поколения технологий, новый цикл роста может обеспечить создание новой фундаментальной революционной технологии, которая сможет ещё сильнее и достаточно заметно повысить производительность труда. Если дело в абсолютном сокращении размера сегмента непроизводственных издержек, новый рост для технологий этого сегмента сможет обеспечить только его абсолютное расширение, которое зависит от общих потребностей в труде такого рода и может быть обеспечено абсолютным расширением остальных сегментов деятельности — материального производства, услуг, создания интеллектуального продукта.

\section*{\raggedright{Пределы роста: производство}}

В том случае, если реализация экономического эффекта в секторе производства идёт по сценарию сокращения рабочей силы с сохранением прежнего объема производства, внедрение новых технологий в сегмент материального производства будет сокращать этот сегмент по количеству вложенных в него ресурсов и задействованной рабочей силы, логика развития событий будет аналогична логике сокращения сегмента непроизводственных издержек (объем производства остаётся прежним, часть рабочей силы отправляется в сегмент незанятых).

В том случае, если реализация эффекта происходит по пути увеличения производства прежними силами, внедрение новых технологий будет приводить к увеличению абсолютной массы богатства общества. Добавочный материальный продукт расширит абсолютные границы всех прочих сегментов мировой фабрики, создаст необходимые условия для роста населения. Пределами роста для реализации эффекта от внедрения технологий в таком случае выступят доступность природных ресурсов, необходимых для производства целевого продукта, а также потребность общества в этом продукте, ограниченная текущим количеством всего населения планеты и темпами его роста.

В том случае, если экономический эффект реализуется по сценарию сокращения рабочего дня, высвободившееся свободное время рабочих не конвертируется напрямую в форму освободившегося или добавочного материального ресурса. Однако в масштабах общества оно обеспечивает добавочный спрос на услуги, создавая необходимые предпосылки для расширения сектора услуг в целом. Кроме того, также пополняет фонд свободного времени досуга, которое по инициативе индивида может быть потрачено на научно"=технические исследования и творчество, плоды которого могут вернуться в общество «бесплатным» «незапланированным» интеллектуальным продуктом.

\section*{\raggedright{Пределы роста: услуги}}

Абсолютный размер сегмента услуг ограничен размером работоспособной части всего населения, не занятой напрямую в сегменте материального производства, а также в сегменте непроизводственных издержек, необходимых для организации нормального течения процесса производства материальных благ. Минимально необходимый размер сегмента услуг диктуется потребностями как сегмента производства, так и всего общества в целом, — всеобщее базовое образование, подготовка специалистов, медицина, пассажирские перевозки и т.~п. Однако все возможные услуги, представленные в сегменте, не ограничены только минимально необходимыми потребностями общества. Разнообразные услуги в области культуры, развлечений, спорта, творчества, путешествий, красоты, дополнительного образования и т.~п. могут появляться и потребляться в том случае, если на них есть достаточный спрос. При повышении производительности труда в сегменте услуг, старая услуга может быть предоставлена большему количеству потребителей при меньших затратах труда, но освободившиеся ресурс и рабочая сила могут быть направлены на предоставление новой услуги для этой же или другой аудитории. Таким образом, в отличие от сегмента непроизводственных издержек, внедрение технологий в сегмент услуг не обязательно ведёт к его сокращению, но, напротив, может вести к увеличению разнообразия предоставляемых услуг и, в целом, к более широкому охвату при сохранении прежнего количества ресурсов и рабочей силы внутри сегмента.

Для того, чтобы обеспечить процесс потребления услуг, кроме направления ресурсов в сегмент услуг для создания инфраструктуры и оплаты рабочей силы специалистов, требуется наличие условий, при которых на услуги будет достаточный спрос. Кроме объективной потребности или субъективного желания получить ту или иную услугу, у потенциального потребителя должно быть выделено достаточное количество времени на ее потребление. Возможность выделения необходимого времени для получения базовых медицинских услуг при условии полной занятости на основном месте работы регулируется законодательством — оплачиваемые больничные, отгулы, плановые медицинские осмотры, диспансеризация и т.~п. Время взросления детей полностью посвящено всестороннему развитию их личности через потребление всевозможных базовых и дополнительных образовательных, спортивных, развлекательных и прочих услуг. Иногда этот период продляется на период получения высшего образования. Получение взрослыми работниками необходимых образовательных услуг таких, как повышение квалификации, может согласовываться с работодателем и зачитываться как оплаченное рабочее время. Для получения прочих услуг, не входящих в список базовых необходимых, работающий индивид сам выделяет время — отпуск, выходные, праздники, отгулы, свободное время перед началом или после завершения рабочего дня.

Таким образом, общая продолжительность рабочего дня во всех сегментах деятельности внутри мировой фабрики прямо влияет на уровень спроса на разнообразные услуги, значит на развитие всего сегмента услуг. Реализация экономических эффектов от внедрения технологий в производство или прочие сегменты через сокращение рабочего дня при выполнении прежнего объема работы будет создавать необходимые условия для роста и развития сегмента услуг. Достаточное количество освободившегося от необходимой работы времени также может привести к тому, что некоторые специалисты в своих областях будут не только потреблять услуги, но и делиться собственным опытом — проводить лекции, создавать кружки, клубы по интересам, т.~е. станут сами оказывать услуги, что, в целом, приведет к ещё большему расширению и развитию сегмента услуг.

\section*{\raggedright{Пределы роста: исследования и разработки}}

Максимальное количество специалистов, занятых в сегменте технологий, как и в случае с сегментом услуг, будет ограничено количеством работников мировой фабрики, не занятых прямо в сегменте материального производства и обеспечивающей его части сегмента непроизводственных издержек. Кроме того, для нормального функционирования сегмента технологий некоторое количество специалистов должно быть занято в сегменте услуг для обеспечения базовых услуг и в системе образования для подготовки инженерных и научных кадров.

Максимальное количество материальных ресурсов, которое может быть направлено в сегмент исследований и разработок на очередном экономическом цикле, ограничено размером сегмента дохода предпринимателей (который может послужить источником для первоначальных инвестиций), а также суммарным размером всех экономических эффектов, которые были получены внедрением технологий, разработанных на прошлом экономическом цикле (реинвестирование). В действительности, распорядители ресурсов, направляя их в сегмент технологий, рассчитывают, что инвестиции текущего цикла разработок, как минимум, будут компенсированы экономическим эффектом на следующем цикле внедрения, в идеале — покажут рост. Поэтому на количество средств, направленных в сегмент технологий, будет также влиять расчетный потенциальный экономический эффект на грядущем цикле внедрения. В том случае, если ожидаемый экономический эффект от текущего поколения технологий велик, сегмент технологий будет притягивать средства инвесторов, общее количество вложений в разработки будет стремиться к своей максимальной границе. В том случае, если текущее поколение технологий себя в значительной степени исчерпало, ожидаемый экономический эффект от новых внедрений не сможет окупить затраты на новые базирующиеся на текущем поколении технологий разработки. При таких обстоятельствах следует ожидать оттока ресурсов из сегмента разработок, вплоть до его сокращения по отношению к размеру, достигнутому на текущем цикле экономики.

Размер экономического эффекта от внедрения технологического продукта не зависит от количества инженеров, которые его разработали. Однако для многих типовых инженерных проектов можно давать достаточно реалистические оценки, какое количество специалистов и какое количество времени потребуется для их реализации. Технологические компании, имея представление о том, какого рода продукты имеют потенциал реализации экономического эффекта и могут быть внедрены достаточно широко, формируя на основе такого представления среднесрочный фронт работ, получают возможность оценивать свои потребности в профильных специалистах.

Таким образом, размер сегмента разработок, нацеленных на внедрение на ближайшем экономическом цикле, определяется точкой оптимального соотношения между расчетным ожидаемым экономическим эффектом от внедрения рассматриваемого массива разработок и количеством ресурсов, необходимых для реализации этих разработок. Ресурсы, выделенные на исследования и разработки ранних стадий с более поздним ожидаемым временем внедрения и реализации эффекта, составят другую часть общего сегмента технологий. Их объем будет определяться тем же отношением, что и в случае с разработками ближайшей перспективы, но с поправкой на более длительный период до получения ожидаемого эффекта. Общий размер сегмента будет ограничен сверху количеством свободных ресурсов общества, которое оно может направить в сегмент технологий.

Количество ресурсов, необходимых для реализации того или иного технологического проекта, зависит от текущего уровня научно"=технического прогресса, в т.~ч. от производительности интеллектуального труда исследователей и инженеров, в значительной степени определяемой инструментами разработки, которые есть в их распоряжении. В том случае, если появляется некоторый технический инструмент, который позволяет решить прежнюю инженерную задачу меньшими затратами труда, т.~е. заметно дешевле, это обстоятельство оказывает влияние на точку оптимальности соотношения между затратами труда на решение задачи и потенциальным экономическим эффектом, который можно получить ее решением. В некоторых ситуациях может оказаться, что экономический эффект от решения задач целевых сегментов мировой фабрики, решение которых было нецелесообразно с прежней производительностью труда инженеров, сможет окупить затраты на их решение, если при разработке применить новый набор инструментов. Таким образом, такое повышение производительности труда инженеров через создание новых инструментов для разработок, которое расширяет поле целесообразных, но ещё не решенных задач, расширяет размер потенциально достижимого экономического эффекта, создаёт условия для расширения сегмента разработок.

По данным портала Internet live stats, количество веб-сайтов увеличилось на 438\% с 3 млн. в 1999 году до 17 млн. в 2000 году \cite{internetlivestatsTotalNumberOfWebsites2018}. По данным компании WebpageFX, нижняя планка обычной ожидаемой цены разработки веб-сайта в это же время опустилась почти в два с половиной раза, с \$6000 в 1998-2000 годы до \$2500 в 2000-2004 \cite{jlbworksTheCostOfWebsiteStats2014}. Вполне вероятно, что заметное снижение цены разработки типового веб-сайта могло расширить аудиторию потенциальных потребителей, которые теперь могли его себе позволить и в рамках текущей деятельности окупить. Понижение цены разработки среди прочих факторов могло быть вызвано появлением новых инструментов в области веб-разработки\footnote{«modern tools have made it much easier for Web designers to create great looking websites in much less time than before» \cite{jlbworksTheCostOfWebsiteStats2014}}. Конечно, на рост количества коммерческих веб-страниц влиял также рост общей аудитории интернета (с 16 млн. пользователей, 0.5\% населения Земли, в 1995 году до 5.5 млрд. пользователей, 69\% населения, в 2022 \cite{internetworldstatsInternetGrowsStats2022}), который делал всё более целесообразными расходы на создания веб-сайта. В 2023 году 71\% субъектов малого бизнеса имеют веб-сайт, в 2024 средняя цена разработки базового сайта составляет \$3200, количество сайтов составляет более миллиарда \cite{forbesWebsiteStats2024}.

Разработки, повышающие производительность труда инженеров, но не настолько, чтобы расширить поле потенциальных проектов, удешевляя и ускоряя процесс решения текущих типовых задач, позволяют реализовать весь потенциальный экономический эффект за более короткий промежуток времени (больше проектов будет реализовано и внедрено в целевые сегменты мировой фабрики за более короткое время). Тем самым на рассматриваемом периоде они приносят добавочный экономический эффект, вместе с тем ускоряют процесс исчерпания текущего поколения фундаментальных технологий. Но даже эволюционные технологии сами по себе не ведут к сокращению сегмента разработок — тенденция к сокращению проявит себя только тогда, когда текущее поколение технологий окажется, действительно, исчерпанным.

Допустим, в распоряжении общества появилась некоторая технология, которая позволяет на своей основе создавать решения автоматизации в некотором целевом сегменте экономики, к примеру, создавать мобильные приложения для клиентов банков по более"=менее стандартному шаблону, но с необходимостью провести часть дополнительной разработки под конкретный банк. Потенциалом рынка для разработчиков таких приложений будут все банки. Экономические эффекты от сокращения издержек на обслуживание клиентов после внедрения приложения в некотором количестве банков составят приращение общего экономического эффекта на текущем цикле внедрения. Весь возможный потенциал приращения эффекта за всё время будет ограничен количеством всех банков вообще, т.~е. всех потенциальных потребителей рассматриваемых разработок. Разработчики приложений такого рода, работая с некоторой установившейся производительностью, смогут создать решения для некоторого количества банков, обеспечивая на следующем за разработкой цикле внедрения соответствующий прирост экономического эффекта (Рис. \ref{fig:pic9_total_effect_norm_productivity}). При неизменном размере сегмента разработок инженеры, занятые реализацией таких проектов, смогут обеспечить всех клиентов целевыми решениями за несколько циклов разработки и внедрения в более-менее равномерном темпе, обеспечивая на каждом цикле внедрения более"=менее одинаковый прирост экономического эффекта. После того, как у всех возможных клиентов решения такого рода будут внедрены, можно считать, что фундаментальная технология себя исчерпала, новые разработки на ее основе не будут востребованы, инженерам, специализировавшимся на разработках такого рода, потребуется переквалифицироваться на разработки другого рода, или сегмент разработок будет сокращен.

\begin{figure}[ht]
    \centering
    \includegraphics[width=0.95\textwidth]{pic9_total-effect-norm-productivity}
    \caption{Разработка и внедрение массивов разработок на базе фундаментальной технологии с обычной производительностью инженеров — реализация обычного экономического эффекта}
    \label{fig:pic9_total_effect_norm_productivity}
\end{figure}

Теперь, если на одном из промежуточных циклов разработки появилась некоторая технология, которая повысит производительность труда инженеров"=разработчиков в два раза (Рис. \ref{fig:pic10_total_effect_higher_productivity}), вдвое большее количество типовых проектов автоматизации может быть реализовано прежними силами за прежнее время или вдвое меньшее количество разработчиков сможет реализовать прежнее количество проектов за прежнее время. До тех пор, пока на рынке все еще присутствуют потенциальные новые клиенты, пока еще не заказавшие разработку и не внедрившие систему, компании"=разработчику нет смысла сокращать команду разработки, т.~к. она, не меняя внутренние процессы и не привлекая дополнительные ресурсы, сможет реализовать проекты для вдвое большего количества клиентов. Сократить разработчиков после повышения их производительности при неисчерпанном целевом рынке — значит отказаться от экономического эффекта, который может принести дополнительная масса разработок. Те клиенты, которые могли получить персональный продукт только на следующем экономическом цикле, смогут получить его уже на ближайшем цикле внедрения. Приращение экономического эффекта на ближайшем цикле внедрения произойдет в удвоенном размере: одну часть приращения составит экономический эффект от внедрения проектов, которые так или иначе были бы разработаны с исходной производительностью инженеров, другая часть (добавочный экономический эффект) — экономический эффект от проектов, которые были реализованы на текущем цикле разработки благодаря увеличившейся производительности инженеров.

Тем не менее, добавочный экономический эффект мог быть получен благодаря тому, что потенциал фундаментальной технологии все еще не был исчерпан в тот момент, когда была повышена производительность труда разработчиков. Его обеспечили клиенты, которые так или иначе собирались заказать и внедрить разработку, но смогли это сделать и реализовать экономический эффект в более ранний срок. Таким образом, повышение производительности труда инженеров"=разработчиков, обеспечивая добавочный экономический эффект с более высокими темпами приращения экономического эффекта, вместе с тем сокращает время исчерпания целевого рынка. После того, как все потенциальные клиенты внедрят у себя целевые разработки, старые внедрения продолжат создавать и накапливать эффект на достигнутом постоянном уровне, но прирост эффекта за счет новых внедрений прекратится, потребность в новых разработках отпадет. Без абсолютного роста целевой части рынка разработчикам придется переквалифицироваться на разработку решений другого рода, если в них возникнет дополнительная потребность, или сегмент разработок будет сокращен.

\begin{figure}[ht]
    \centering
    \includegraphics[width=0.95\textwidth]{pic10_total-effect-higher-productivity}
    \caption{Разработка и внедрение массивов разработок на базе фундаментальной технологии с повышением производительности инженеров — реализация добавочного экономического эффекта}
    \label{fig:pic10_total_effect_higher_productivity}
\end{figure}

В том случае, если заметное сокращение сегмента разработок совпадет с заметным повышением производительности инженерного труда, может создаться видимость, что причиной сокращения является новый инструмент, повысивший производительность разработок, но действительной причиной будет исчерпание целевого сегмента\footnote{«The ultimate blame for the collapse, according to Lewis, is that Convoy found itself in “the middle of a massive freight recession and a contraction in the capital markets.”» \cite{freightwavesConvoyShutting2023}}, на который была направлена значительная доля новых технологических проектов. В том случае, если повышение производительности инженерного труда удешевит процесс разработки настолько, что база потенциальных потребителей технологии будет расширена, повышение производительности разработки такого рода может замедлить процесс сокращения сектора технологий или привести к его расширению.

Таким образом, размер сегмента технологий является производным от ожидаемого экономического эффекта, который может быть получен в прочих целевых сегментах мировой фабрики. Рынок программных продуктов для разработчиков программных продуктов — производная от части сегмента технологий, представляющей производство программных продуктов прямого внедрения.

% чтобы ссылки в тексте имели правильные номера, генератор нужно запустить два раза
% нижние подчеркивания '_' в ссылках заменяем на '\_', иначе они воспринимаются как спец-символы, если они не обернуты в \url{}
% если ссылка обернута в \url{}, нижние подчеркивания экранировать не надо
\begin{thebibliography}{2}
\bibitem{ecoEffects} Экономические эффекты автоматизации с ИТ / А.~Е.~Моисеев // на правах рукописи, – 2024. [Электронный ресурс]. – URL: \url{https://github.com/sadr0b0t/it-automata-part1}
\bibitem{businessModels} Экономические модели распространения цифровых продуктов / А.~Е.~Моисеев // на правах рукописи, – 2024. [Электронный ресурс]. – URL: \url{https://github.com/sadr0b0t/it-automata-part2}
\bibitem{disserNagorniDigitalTransformWorld2021} Нагорный, Д.~А. Цифровая трансформация мировой экономики: тенденции и перспективы : специальность 08.00.14 "Мировая экономика" : диссертация на соискание ученой степени кандидата экономических наук / Нагорный Дмитрий Александрович, 2021. – 212 с. – EDN MGZKNA.
\bibitem{einsteinInventor1982} Эйнштейн-изобретатель / В.~Я.~Френкель, Б.~Е.~Явелов // Наука. – 1982. – С. 140.
\bibitem{businesscomBigProfitsBigLayoffs2013} Big Profits, Big Layoffs: Tracking Earnings Per Laid off Employee / Chad Brooks // business.com, – 2023. [Электронный ресурс]. – URL: \url{https://www.business.com/finance/big-tech-earnings-and-layoffs-compared/}
\bibitem{wallstreetWeWorkCashed2019} WeWork Co-Founder Has Cashed Out at Least \$700 Million Via Sales, Loans / Eliot Brown, Maureen Farrell, Anupreeta Das // The Wall Street Journal, – 2019. [Электронный ресурс]. – URL: \url{https://www.wsj.com/articles/wework-co-founder-has-cashed-out-at-least-700-million-from-the-company-11563481395}
\bibitem{forbesWeWorkTimeline2023} WeWork’s Rise To \$47 Billion—And Fall To Bankruptcy: A Timeline / Britney Nguyen // The Forbes, – 2023. [Электронный ресурс]. – URL: \url{https://www.forbes.com/sites/britneynguyen/2023/11/07/weworks-rise-to-47-billion-and-fall-to-bankruptcy-a-timeline/}
\bibitem{agileSoftDevel2002} Agile Software Development Ecosystems / Jim Highsmith // Addison Wesley. – 2002. – С. 30-31. – eBook ISBN 0-201-76043-6
\bibitem{rbcAutodeskForbidsRussia2024} Autodesk запретила российским компаниям использовать свой софт / Екатерина Шокурова // РБК, – 2024. [Электронный ресурс]. – URL: \url{https://www.rbc.ru/technology_and_media/22/03/2024/65fd84b09a7947213a06d174}
\bibitem{rbcMicrosoftCancelCloudRussia2024} Microsoft закроет доступ к облачным сервисам для компаний в России / Екатерина Ясакова, Елена Чернышова // РБК, – 2024. [Электронный ресурс]. – URL: \url{https://www.rbc.ru/technology_and_media/15/03/2024/65f452069a79470ded3adf19}
\bibitem{tadviserGazprombankSubstImport2024} Александр Егоркин, Газпромбанк: Если мы немедленно проведем импортозамещение, бизнес-процессы встанут // TAdviser, – 2024. [Электронный ресурс]. – URL: \url{https://www.tadviser.ru/a/797630}
\bibitem{microsoft10K2006} MICROSOFT CORPORATION, Form 10-K ANNUAL REPORT PURSUANT TO SECTION 13 OR 15(d) OF THE SECURITIES EXCHANGE ACT OF 1934 FOR THE FISCAL YEAR ENDED JUNE 30, 2006 // U.S. Securities and Exchange Commission, – 2006. [Электронный ресурс]. – URL: \url{https://www.sec.gov/Archives/edgar/data/789019/000119312506180008/d10k.htm}
\bibitem{winsupersiteWindowsXP2001} Windows XP: The Road to Gold / Paul Thurrott // Paul Thurrott's SuperSite for Windows, – 2001. [Электронный ресурс]. – URL: \url{https://web.archive.org/web/20010828175222/http://www.winsupersite.com/reviews/windowsxp_gold.asp}
\bibitem{microsoft10K2001} MICROSOFT CORPORATION, Form 10-K ANNUAL REPORT PURSUANT TO SECTION 13 OR 15(d) OF THE SECURITIES EXCHANGE ACT OF 1934 FOR THE FISCAL YEAR ENDED JUNE 30, 2001 // U.S. Securities and Exchange Commission, – 2001. [Электронный ресурс]. – URL: \url{https://www.sec.gov/Archives/edgar/data/789019/000103221001501099/d10k.txt}
\bibitem{microsoft10K2003} MICROSOFT CORPORATION, Form 10-K ANNUAL REPORT PURSUANT TO SECTION 13 OR 15(d) OF THE SECURITIES EXCHANGE ACT OF 1934 FOR THE FISCAL YEAR ENDED JUNE 30, 2003 // U.S. Securities and Exchange Commission, – 2003. [Электронный ресурс]. – URL: \url{https://www.sec.gov/Archives/edgar/data/789019/000119312503045632/d10k.htm}
\bibitem{microsoftSecProfile} MICROSOFT CORP MSFT on Nasdaq // U.S. Securities and Exchange Commission. [Электронный ресурс]. – URL: \url{https://www.sec.gov/edgar/browse/?CIK=789019}
\bibitem{nyTimesUsSuesAdobe2024} U.S. Sues Adobe Over Hard-to-Cancel Subscriptions / David McCabe // The New York Times, – 2024. [Электронный ресурс]. – URL: \url{https://www.nytimes.com/2024/06/17/technology/us-adobe-subscription-lawsuit.html}
\bibitem{nyTimesInRaceToBuildAi2024} In Race to Build A.I., Tech Plans a Big Plumbing Upgrade / Karen Weise // The New York Times, – 2024. [Электронный ресурс]. – URL: \url{https://www.nytimes.com/2024/04/27/technology/ai-big-tech-spending.html}
\bibitem{lwnLinuxKernelDevelStats2022} Development statistics for the 6.1 kernel (and beyond) / Jonathan Corbet // LWN.net, – 2022. [Электронный ресурс]. – URL: \url{https://lwn.net/Articles/915435/}
\bibitem{carlschwanKdeCommunity2023} Health of the KDE community (2023 Update) / Carl Schwan // carlschwan.eu, – 2023. [Электронный ресурс]. – URL: \url{https://carlschwan.eu/2023/04/28/health-of-the-kde-community-2023-update/}
\bibitem{hostingadviceKdeCommunity2017} How KDE’s Vast Open-Source Community Has Been Developing Technologies to Bring Reliable, Monopoly-Free Computing to the World for 20+ Years / Sean Garrity // HostingAdvice.com, – 2017. [Электронный ресурс]. – URL: \url{https://www.hostingadvice.com/blog/kde-community-delivers-open-source-monopoly-free-computing/}
\bibitem{kdenewsKdabLicense2012} Provident License Agreements for KDAB // KDE.news, – 2012. [Электронный ресурс]. – URL: \url{https://dot.kde.org/2012/04/21/provident-license-agreements-kdab}
\bibitem{thevergeValveSteamDeck2022} Valve answers our burning Steam Deck questions — including a possible Steam Controller 2 / Sean Hollister // The Verge, – 2022. [Электронный ресурс]. – URL: \url{https://www.theverge.com/23499215/valve-steam-deck-interview-late-2022}
\bibitem{phoronixValveIsFundingKde2019} Valve Is Funding Improvements To KDE's KWin \& More Work On X.Org / Michael Larabel // Phoronix, – 2019. [Электронный ресурс]. – URL: \url{https://www.phoronix.com/news/Valve-Funding-KWin-Work}
\bibitem{kdeevReport2022} KDE e.V. COMMUNITY REPORT, 2022, Issue 39 // KDE e.V., – 2023. [Электронный ресурс]. – URL: \url{https://ev.kde.org/reports/ev-2022/}
\bibitem{githubOpenSourseSurvey2017} Open Source Survey // GitHub, – 2017. [Электронный ресурс]. – URL: \url{https://opensourcesurvey.org/2017/}
\bibitem{nplus1OpensourseRussia2024} Исследование: опенсорс в России // N + 1, – 2024. [Электронный ресурс]. – URL: \url{https://research.nplus1.ru}
\bibitem{businessinsiderGoogle20percentTruth2015} The truth about Google's famous '20\% time' policy / Jillian D'Onfro // Business Insider, – 2015. [Электронный ресурс]. – URL: \url{https://www.businessinsider.com/google-20-percent-time-policy-2015-4}
\bibitem{qzGoogle20percentDead2013} Google’s “20\% time,” which brought you Gmail and AdSense, is now as good as dead / Christopher Mims // Quartz, – 2013. [Электронный ресурс]. – URL: \url{https://qz.com/115831/googles-20-time-which-brought-you-gmail-and-adsense-is-now-as-good-as-dead}
\bibitem{fastcompanySilliconValleyGarageMyth2018} The origins of Silicon Valley’s garage myth / Olivia Erlanger, Luis Ortega Govela // Fast Company, – 2018. [Электронный ресурс]. – URL: \url{https://www.fastcompany.com/90270226/the-origins-of-silicon-valleys-garage-myth}
\bibitem{dilbertGarage1} Dilbert comic strip, Friday May 12, 2006 / Scott Adams // dilbert.com, – 2006. [Электронный ресурс]. – URL: \url{https://web.archive.org/web/20230305075250/https://dilbert.com/strip/2006-05-12}
\bibitem{dilbertGarage2} Dilbert comic strip, Friday May 19, 2006 / Scott Adams // dilbert.com, – 2006. [Электронный ресурс]. – URL: \url{https://web.archive.org/web/20230305164931/https://dilbert.com/strip/2006-05-19}
\bibitem{stanfordOriginalGoogleComputer1996} The Original GOOGLE Computer Storage [Page and Brin] (1996) // Stanford University InfoLab, – [Электронный ресурс]. – URL: \url{http://infolab.stanford.edu/pub/voy/museum/pictures/display/0-4-Google.htm}
\bibitem{businessinsiderAppleGarageMyth2014} Apple Cofounder Says The Famous Garage Where He Started Apple With Steve Jobs Is 'A Myth' / Lisa Eadicicco // Business Insider, – 2014. [Электронный ресурс]. – URL: \url{https://www.businessinsider.com/apple-co-founder-steve-wozniak-famous-garage-is-a-myth-2014-12}
\bibitem{sociologySociologicalKDE2011} Challenging Code: A Sociological Reading of the KDE Free Software Project / Brian Alleyne // Sociology. – 2011. – Volume. 45, Issue 3. – P. 496–511. – DOI 10.1177/0038038511399620.
\bibitem{zolotovManufacturingLabour2006} Золотов, А.~В. Философия производительного труда : монография / А.~В.~Золотов, М.~В.~Попов ; А.~В.~Золотов, М.~В.~Попов; Федеральное агентство по образованию, Нижегородский гос. ун-т им. Н.~И.~Лобачевского. – Нижний Новгород : Изд-во Нижегородского гос. ун-та им. Н.~И.~Лобачевского, 2006. – 158 с. – ISBN 5-85746-897-3. – EDN QRGYUB.
\bibitem{internetlivestatsTotalNumberOfWebsites2018} Total number of Websites // Internet live stats, – 2018. [Электронный ресурс]. – URL: \url{https://www.internetlivestats.com/total-number-of-websites/}
\bibitem{jlbworksTheCostOfWebsiteStats2014} The Cost of a Website From 1995 to the Present // webpagefx.com, – 2014. [Электронный ресурс]. – URL: \url{https://jlbworks.com/web-design/the-cost-of-a-website-from-1995-to-the-present/}
\bibitem{internetworldstatsInternetGrowsStats2022} Internet growth statictics // Internet World Stats, – 2022. [Электронный ресурс]. – URL: \url{https://web.archive.org/web/20240526212800/https://www.internetworldstats.com/emarketing.htm}
\bibitem{forbesWebsiteStats2024} Top Website Statistics For 2024 / Katherine Haan, Rob Watts // Forbes, – 2024. [Электронный ресурс]. – URL: \url{https://web.archive.org/web/20240530215121/https://www.forbes.com/advisor/business/software/website-statistics/}
\bibitem{freightwavesConvoyShutting2023} It’s over: Convoy shutting operations, no strategic white knight to the rescue / John Kingston // FreightWaves, – 2023. [Электронный ресурс]. – URL: \url{https://www.freightwaves.com/news/its-over-convoy-shutting-operations-no-strategic-white-knight-to-the-rescue}
\end{thebibliography} 

\end{document}

